% ************************** Thesis Abstract *****************************
% Use `abstract' as an option in the document class to print only the titlepage and the abstract.
\begin{abstract}

Epigenetic reprogramming is an essential part of reproduction across both animal and plant kingdoms. Although the two kingdoms have diverged more than 1.5 billion years ago, they both have convergently evolved systems that protect the integrity of the genome throughout generations. This is in part due to the threat of mobile transposable elements, which when uncontrolled can lead to genomic instability through insertions and rearrangements. Whilst animals reset DNA methylation during germline development, plants largely maintain symmetric CG and CHG DNA methylation. 

However, asymmetric CHH methylation is reprogrammed in a tightly regulated manner during the development of the male sexual lineage in Arabidopsis thaliana. This is accomplished through germline-specific, non-canonical RNA-directed DNA methylation, which transitions from CHH hypomethylation to CHH methylation while also introducing novel local CHH hypermethylation that is specific to the germline. Whilst the discovery of nurse-cell specific sRNAs and mismatch targeting in the meiocyte have contributed to the understanding of part of this pathway, the specific sRNA profiles and the mechanisms governing the establishment and maintenance of DNA methylation patterns in other germline cells, such as the microspore and pollen, remain elusive.

Other germline-specific epigenetic modifications have also been described including the methylation of cytosine on the 4th carbon: N4-methylcytosine (4mC). 4mC plays an important role in host defence in prokaryotic organisms but more recently has been found in the male germline cells of Marchantia polymorpha. As the mature embryo loses this epigenetic mark, it’s important for our understanding to see whether this epigenetic mark is maintained or passively lost after fertilisation.

\end{abstract}
