% ************************** Thesis Abstract *****************************

\begin{abstract}

Epigenetic reprogramming is an essential part of reproduction across both animal and plant kingdoms. Although the two kingdoms have diverged more than 1.5 billion years ago, they both have convergently evolved systems that protect the integrity of the genome throughout generations. This is in part due to the threat of mobile transposable elements, which when uncontrolled can lead to genomic instability through insertions and rearrangements. Whilst animals reset DNA methylation during germline development, plants largely maintain symmetric CG and CHG DNA methylation. However, asymmetric CHH methylation undergoes tightly regulated reprogramming during male sexual lineage development in plants.

This thesis investigates the dynamics of epigenetic regulation during plant reproduction in two distinct systems: \textit{Arabidopsis thaliana} and \textit{Marchantia polymorpha}. In \textit{Arabidopsis}, the expression patterns of key RNA-directed DNA methylation (RdDM) components was examined, alongside exploring the small RNA (sRNA) profiles post-meiosis during the development of the male sexual lineage. The results reveal a complex spatiotemporal regulation of RdDM factors and a shift away from CLSY-dependent sRNA biogenesis after meiosis, potentially involving non-canonical RdDM pathways.

In the basal land plant \textit{Marchantia polymorpha}, recent studies have revealed the presence of N4-methylcytosine (4mC) in the male germline, an epigenetic modification previously thought to exist only in prokaryotes. During spermiogenesis, 4mC is deposited globally at CG sites, yet after fertilisation, the mature embryo loses this mark, and the knockout of 4mC leads to accelerated embryonic development. This study further explores the fate of paternal 4mC in early \textit{Marchantia} embryos and provides evidence that 4mC is actively removed before pronuclear fusion. Further investigation reveals that the loss of 4mC accelerates the onset of the first nuclear division and imposes a partial fertility cost, ultimately impacting long-term embryonic development.

\end{abstract}

