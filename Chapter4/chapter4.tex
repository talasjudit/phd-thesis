%!TEX root = ../thesis.tex
%*******************************************************************************
%****************************** Fourth Chapter **********************************
%*******************************************************************************
\chapter{Main Discussion}

% **************************** Define Graphics Path **************************
\ifpdf
    \graphicspath{{Chapter4/Figs/}}
\else
    \graphicspath{{Chapter3/Figs/}}
\fi


This thesis aimed to investigate the epigenetic mechanisms regulating gene expression and genome stability during plant reproduction, focusing on two plant systems: \textit{Arabidopsis thaliana} and \textit{Marchantia polymorpha}. Specifically, I examined the expression patterns of key RNA-directed DNA methylation (RdDM) components in \textit{Arabidopsis} the development of the male sexual lineage, described the 24nt sRNA profiles after meiosis in microspores and sperm. I also explored the dynamics of DNA methylation, particularly the inheritance of the paternal epigenetic mark N4-methylcytosine (4mC), during early embryo development in \textit{Marchantia}. To address these aims, I employed a combination of small RNA sequencing, bisulfite and APOBEC3A-mediated methylation sequencing,  fluorescent reporter line construction and imaging and developing bioinformatics pipelines and data analysis of novel and existing data. The findings reveal novel insights into the epigenetic regulation of plant reproduction and highlight the diverse mechanisms employed by different plant lineages.

\section{Tissue-Specific Regulation of RdDM}

The analysis of NRPD1 and NRPE1 expression patterns in \textit{Arabidopsis} revealed a dynamic regulation of these key RdDM components during pollen development. NRPD1, the largest subunit of RNA Polymerase IV, showed confined expression in the tapetal layer and surrounding somatic anther tissue during the meiocyte stage, with subsequent expression in microspores but absence in mature pollen. In contrast, NRPE1, the largest subunit of RNA Polymerase V, exhibited broader expression in both somatic and germline tissues throughout anther development. 

Investigation of CLSY chromatin remodeler expression in anthers provided further insights into RdDM regulation. CLSY3 expression was confirmed to be confined to the tapetum and absent in later developmental stages. Similarly, CLSY1 and CLSY2 were not detected in maturing pollen. However, some discrepancies were noted when these findings were cross-referenced with RNA-sequencing data. Despite these ambiguities, the sRNA profiles largely corroborated the observed expression patterns of CLSY proteins. This suggests a potential shift in the regulation of sRNA production after meiosis, possibly involving alternative mechanisms for small RNA biogenesis and highlighting the importance of non-canonical RdDM pathways in the male sexual lineage.

These findings collectively indicate a complex temporal and spatial regulation of the RdDM pathway during the development of the male sexual lineage, with potential implications for the establishment and maintenance of DNA methylation patterns. The observed changes in CLSY expression patterns may reflect the dynamic nature of epigenetic regulation during pollen development and highlight the need for further investigation into the mechanisms governing sRNA production in mature pollen.

\section{Sperm cell-specific small RNA clusters are uncoupled from CLSY-dependent loci}

The analysis of small RNA profiles in \textit{Arabidopsis} sperm cells revealed a unique landscape distinct from both earlier germline stages and somatic tissues. Firstly, a family of LTR/Gypsy TEs was identified as producing sperm-specific 24nt sRNAs which may hint at ATGP2N reactivation in the sperm cell, or the 24nt sRNAs may originate from the vegetative cell based on CHH profiles. Secondly, around 20\% of MetGenes produce 24nt sRNAs specifically in the sperm cell, including genes important in mediating CG methylation. Finally, while CLSY1 and CLSY2-dependent loci produce abundant sRNAs in somatic tissues, sperm cells exhibit a different pattern. Many sperm-specific small RNA clusters do not overlap with known CLSY-dependent loci, further hinting at the involvement of non-canonical RdDM pathways in the biogenesis or sRNAs in this tissue. These findings indicates a shift in the regulation of epigenetic pathways during the final stages of the development of the male sexual lineage, which may be crucial for genome integrity. 

\section{A novel dissection and sequencing protocol showed that paternal 4mC is lost in early \textit{Marchantia} embryos and that the knockout of paternal 4mC results in accelerated early embryonic development}

A novel method of dissecting early \textit{Marchantia} embryos was trialled and perfected, allowing the successful trial of low input AMD and bisulfite sequencing protocols. Furthermore, a fixing and clearing protocol was optimised to explore early embryonic development in Marchantia. Nuclear reporter lines of wild type male, female and \textit{dn4mt1} male \textit{Marchantia} plants were constructed screened and used to study the effect of 4mC knockouts in early embryonic development.

I presented evidence that  paternal 4mC methylation, which is deposited during spermiogenesis, is lost in the early embryo. Our analysis of early embryos (7-8 days after fertilisation) showed no detectable levels of 4mC methylation. This, together with the accelerated developmental phenotype of embryos fertilised by \textit{dn4mt1} mutant sperm hints at the requirement of 4mC removal prior to pronuclear fusion.

Our live cell imaging experiments revealed that embryos fertilised by sperm lacking 4mC methylation (\textit{dn4mt1} mutants) initiate cell division earlier and develop more rapidly during the first 7 days post-fertilization compared to wild-type embryos. This finding provides insight into the functional significance of 4mC in regulating the timing of early embryonic events in Marchantia.

\section{Sporophyte-specific genic methylation is likely targeted by RdDM and a novel method for assigning reads from parental origins for sRNA data is developed}

Regions with \textit{de novo} genic methylation in the sporophyte of \textit{Marchantia} were identified and their relationship with TE-derived sRNA clusters was explored. While sequence homology between TEs and MetGenes was identified, no clear relationship was established between sRNA mismatch targeting and increased CHH methylation at MetGenes. Mapping sRNAs with up to three mismatches did not result in a significant increase in sRNA abundance at these MetGene loci.

The study also explored the potential differences in sRNA production between the paternal and maternal genomes in the sporophyte, especially given the shutdown of the paternal genome. Initial data suggested a paternal bias in sRNA production during the embryo stage, which shifted towards a maternal bias in the sporophyte, although both parental genomes continued to produce substantial amounts of sRNAs.

\section{Future Directions}

\subsection{Understanding cross-cellular transport of sRNAs}

This thesis raises important questions about the mechanisms underlying cross-cellular sRNA movement in reproductive tissues. It seems that the production of sRNAs in somatic tissues adjacent to germline cells with plasmodesmatal or other cytoplasmic connections seems to be a common feature in several plant sexual developmental pathways \citep{RN187,RN293,RN57,RN235}. While it is clear that plasmodesmata facilitate the transfer of sRNAs between the tapetum and meiocytes in \textit{Arabisopsis}, the exact molecular players involved in this process are still poorly understood. Future studies should focus on elucidating the pathways that regulate sRNA movement during germline development.

\subsection{Investigating non-canonical RdDM pathways post-meiosis}

Our discovery of unique small RNA clusters in \textit{Arabidopsis} sperm cells that are uncoupled from known CLSY-dependent loci opens up new avenues for research. Future studies should aim to identify the mechanisms responsible for generating these sperm-specific small RNAs and their potential targets. Investigating the functional significance of these small RNAs in sperm cell development, fertilisation, and early embryogenesis would provide valuable insights into male sexual lineage-specific epigenetic regulation.

\subsection{Investigating the mechanisms of paternal 4mC removal in Marchantia and the embryonic epigenetic landscape}

While we have observed the rapid loss of paternal 4mC in early Marchantia embryos, the exact mechanism remains unclear. Future research should focus on determining whether this loss occurs through passive dilution or active removal. Developing single-cell AMD sequencing methods for zygotes or 2-4 cell stage embryos would provide valuable insights into the dynamics of 4mC loss. Additionally, investigating the potential role of DNA demethylases, particularly MpROS1x, in active 4mC removal would be crucial. Functional studies of MpROS1x, including its potential N4-methylcytosine glycosylase activity, could shed light on the mechanisms of epigenetic reprogramming in early Marchantia embryos.

The accelerated development observed in embryos fertilised by 4mC-deficient sperm suggests a regulatory role for this epigenetic mark in early embryogenesis. Future research should focus on elucidating the molecular mechanisms by which 4mC influences cell cycle progression and gene expression in early embryos. This could involve transcriptome analysis of wild-type and \textit{dn4mt1} mutant embryos at various early developmental stages, as well as identifying potential 4mC readers that may influence its effects on embryonic development. Additionally, investigating the long-term consequences of accelerated early development on sporophyte growth and fitness would provide insights into the evolutionary significance of 4mC-mediated embryonic development. 

Further work should be done investigating both the significance of and the mechanism by which sporophytic \textit{de novo} methylation is established in \textit{Marchantia}. This could include examining RdDM-associated protein variants specific to the sporophyte and employing CRISPR/Cas9 to mutate homologous sequences from potential TE sRNA sources which would help determine whether CHH methylation at the corresponding MetGenes is abolished.

\section{Concluding remarks}

To conclude,  this thesis has uncovered novel aspects of epigenetic regulation during plant reproduction in both \textit{Arabidopsis} and \textit{Marchantia}. It highlights the increasing complexity and dynamic nature of epigenetic reprogramming during germline and embryonic development, conserved across basal land plants and flowering plants. These findings offer new insight into our understanding of plant reproductive epigenetics and hope that it will form the foundation of further studies.